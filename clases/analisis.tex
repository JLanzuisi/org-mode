% Created 2020-12-01 mar 21:54
% Intended LaTeX compiler: pdflatex
\documentclass[11pt]{article}
\usepackage[utf8]{inputenc}
\usepackage[T1]{fontenc}
\usepackage{graphicx}
\usepackage{grffile}
\usepackage{longtable}
\usepackage{wrapfig}
\usepackage{rotating}
\usepackage[normalem]{ulem}
\usepackage{amsmath}
\usepackage{textcomp}
\usepackage{amssymb}
\usepackage{capt-of}
\usepackage{hyperref}
\author{Jhonny Lanzuisi}
\date{\today}
\title{Análisis}
\hypersetup{
 pdfauthor={Jhonny Lanzuisi},
 pdftitle={Análisis},
 pdfkeywords={},
 pdfsubject={},
 pdfcreator={Emacs 26.3 (Org mode 9.1.9)}, 
 pdflang={Spanish}}
\begin{document}

\maketitle
\setcounter{tocdepth}{2}
\tableofcontents


\section{Conjuntos de números}
\label{sec:org9000861}

Este curso se centra en el análisis del sistema de los números reales.
Por lo tanto es esencial conocer \emph{que es} el conjunto de los números reales de forma rigurosa.

\begin{definition}
Sea \(A\) un conjunto ordenado y \(X\subset A\).
\end{definition}

Una prueba de cosas:
\[ \int_{\alpha}\frac{x^{^{2}}}{2} = \sum_{x}^{k} x_{k} -\phi \]

\subsection{{\bfseries\sffamily TODO} Investigar algo de prueba pa ver}
\label{sec:orge39803e}
\end{document}